\documentclass{article}

% if you need to pass options to natbib, use, e.g.:
% \PassOptionsToPackage{numbers, compress}{natbib}
% before loading nips_2017
%
% to avoid loading the natbib package, add option nonatbib:
% \usepackage[nonatbib]{nips_2017}

\usepackage[final]{nips_2017}

% to compile a camera-ready version, add the [final] option, e.g.:
% \usepackage[final]{nips_2017}

\usepackage[utf8]{inputenc} % allow utf-8 input
\usepackage[T1]{fontenc}    % use 8-bit T1 fonts
\usepackage{hyperref}       % hyperlinks
\usepackage{url}            % simple URL typesetting
\usepackage{booktabs}       % professional-quality tables
\usepackage{amsfonts}       % blackboard math symbols
\usepackage{nicefrac}       % compact symbols for 1/2, etc.
\usepackage{microtype}      % microtypography

\title{Pitfalls Encountered When Implementing Complex Neural Networks Like Progressive Neural Networks}

% The \author macro works with any number of authors. There are two
% commands used to separate the names and addresses of multiple
% authors: \And and \AND.
%
% Using \And between authors leaves it to LaTeX to determine where to
% break the lines. Using \AND forces a line break at that point. So,
% if LaTeX puts 3 of 4 authors names on the first line, and the last
% on the second line, try using \AND instead of \And before the third
% author name.

\author{
  Dennis Verheijden\\
  s4455770
  \And
  Joost Besseling\\
  s4796799
  %% examples of more authors
  %% \And
  %% Coauthor \\
  %% Affiliation \\
  %% Address \\
  %% \texttt{email} \\
  %% \AND
  %% Coauthor \\
  %% Affiliation \\
  %% Address \\
  %% \texttt{email} \\
  %% \And
  %% Coauthor \\
  %% Affiliation \\
  %% Address \\
  %% \texttt{email} \\
  %% \And
  %% Coauthor \\
  %% Affiliation \\
  %% Address \\
  %% \texttt{email} \\
}

\begin{document}
 %\nipsfinalcopy is no longer used

\maketitle

\begin{abstract}
  Abstract goes here
\end{abstract}

\section{Introduction}
Vertellen over atari, recent advances. Feel krijgen over hoe complex door zelf implementeren w/e.

The Atari Gym \cite{1606.01540} environment is a modern environment to train and test various reinforcement learning algorithms, on a difficult, real time, task. The gym has a large variety of different games that we can test our algorithms on. We wanted to implement and train a novel neural network in Chainer \cite{chainer_learningsys2015}. We tried implementing a Progressive Neural Network, as proposed by \cite{rusu2016progressive}. 

Unfortunately, we encountered some difficulties when implementing the network in Chainer. In this document we will give an outline of what went wrong, and what went right, so researcher know what problems they should avoid when using Chainer % using chainer is the problem

We also implemented a simpler neural network to ensure ourselves that this was possible in Chainer.

\section{Background}
model beschrijven in termen van \emph{states rewards en actions} + Q-learning (is pure value iteration aka rewards propagaten van de terminal state naar de eerste zet in episodes)

\section{Related Work}
Not sure... Misschien wat vertellen over de guru paper \cite{mnih2013playing} + mogelijke verbeteringen in NATURE \cite{mnih2015human}

\section{Deep Q-Learning}

\subsection{Preprocessing}
Hoe we data eerst verwerken voor complexiteit vermindering + wat input model is (stacked frames) zie \cite{mnih2013playing}

\subsection{Reward Clipping}
Envs hebben andere manier van reward, wij $[-1, 0, 1]$

\subsection{Replay Memory}
Om patronen te voorkomen \cite{mnih2013playing}

\subsection{Training details}
model initializatie + parameters etc. (qua hidden layers e.d.)
Huber loss vs. MSE


\section{Results}


\section{Conclusion}


\section{Discussion}
Complexer dan we dachten + link naar PNN

\cite{*}
\bibliography{bibliography}
\bibliographystyle{plainnat}

\end{document}
