\documentclass{article}

% if you need to pass options to natbib, use, e.g.:
% \PassOptionsToPackage{numbers, compress}{natbib}
% before loading nips_2017
%
% to avoid loading the natbib package, add option nonatbib:
% \usepackage[nonatbib]{nips_2017}

\usepackage[final]{nips_2017}

% to compile a camera-ready version, add the [final] option, e.g.:
% \usepackage[final]{nips_2017}

\usepackage[utf8]{inputenc} % allow utf-8 input
\usepackage[T1]{fontenc}    % use 8-bit T1 fonts
\usepackage{hyperref}       % hyperlinks
\usepackage{url}            % simple URL typesetting
\usepackage{booktabs}       % professional-quality tables
\usepackage{amsfonts}       % blackboard math symbols
\usepackage{nicefrac}       % compact symbols for 1/2, etc.
\usepackage{microtype}      % microtypography
\usepackage{algorithm}
\usepackage{algorithmic}

\title{Pitfalls Encountered When Implementing Complex Neural Networks Like Progressive Neural Networks}

% The \author macro works with any number of authors. There are two
% commands used to separate the names and addresses of multiple
% authors: \And and \AND.
%
% Using \And between authors leaves it to LaTeX to determine where to
% break the lines. Using \AND forces a line break at that point. So,
% if LaTeX puts 3 of 4 authors names on the first line, and the last
% on the second line, try using \AND instead of \And before the third
% author name.

\author{
  Dennis Verheijden\\
  s4455770
  \And
  Joost Besseling\\
  s4796799
  %% examples of more authors
  %% \And
  %% Coauthor \\
  %% Affiliation \\
  %% Address \\
  %% \texttt{email} \\
  %% \AND
  %% Coauthor \\
  %% Affiliation \\
  %% Address \\
  %% \texttt{email} \\
  %% \And
  %% Coauthor \\
  %% Affiliation \\
  %% Address \\
  %% \texttt{email} \\
  %% \And
  %% Coauthor \\
  %% Affiliation \\
  %% Address \\
  %% \texttt{email} \\
}

\begin{document}
 %\nipsfinalcopy is no longer used

\maketitle

\begin{abstract}
  Abstract goes here
\end{abstract}

\section{Introduction}
Vertellen over atari, recent advances. Feel krijgen over hoe complex door zelf implementeren w/e.

The Atari Gym \cite{1606.01540} environment is a modern environment to train and test various reinforcement learning algorithms, on a difficult, real time, task. The gym has a large variety of different games that we can test our algorithms on. We wanted to implement and train a novel neural network in Chainer \cite{chainer_learningsys2015}. We tried implementing a Progressive Neural Network, as proposed by \cite{rusu2016progressive}. 

Unfortunately, we encountered some difficulties when implementing the network in Chainer. In this document we will give an outline of what went wrong, and what went right, so researcher know what problems they should avoid when using Chainer % using chainer is the problem

Finally, we implemented a simpler neural network architecture. We will discuss the results that this network achieved. 

\section{Background}

\subsection{Network Architecture}

model beschrijven in termen van \emph{states rewards and actions} + Q-learning (is pure value iteration aka rewards propagaten van de terminal state naar de eerste zet in episodes)

\subsection{Atari}

The Atari environment uses a default setup for all the games available. First it has to be initialized, after that we can iterate through the screen by calling the $next$ function on the environment with an action. The network will return the next state, the reward, and whether the game is finished or not. When the game is finished, we know we have to either reset the environment, or we are done with the training or evaluation on the game.

\section{Related Work}

Not sure... Misschien wat vertellen over de guru paper \cite{mnih2013playing} + mogelijke verbeteringen in NATURE \cite{mnih2015human}

Reinforcement learning is a very active field, so there is a lot of related work. In 2013 Mnih et al introduced a novel way to tackle the problem of reinforcement learning which they called Q-learning. This method is highly effective, achieving super human play in a number of games.

\section{Deep Q-Learning}

\subsection{Preprocessing}
Hoe we data eerst verwerken voor complexiteit vermindering + wat input model is (stacked frames) zie \cite{mnih2013playing}

Due to the complex nature of raw pixel data, it is a good idea to do a manual preprocessing steps on the data. Some of the difficulties of the raw pixel data: There are a lot of unused pixels in the game, for example the top of Pong only contains the score, so there is no useful information on this part of the screen. We can remove this from the input of the network, to reduce the complexity of the game. Another example is the fact that there is coloured data, the initial input exist of three (i.e. red, green, and blue) channels. By combining them into one input layer, we reduce the complexity even more, without throwing away necessary information.

Another problem with Atari games is that one screen doesn't provide enough information to be able to infer the state of the game (e.g. what direction is the ball moving toward? What direction is the opponent going? Or in pacman, what directions are the ghosts moving in?). In order to mitigate this effect, first we used the difference between the previous screen and the next screen. But since this doesn't provide that much extra information, we decided to use another technique in which we stack the last $n$ frames \cite{mnih2013playing}. Essentialy, we introduce 4 channels, in which each channel contains an entire screen of the game. The idea is that the combination of the last four frames contain enough information to be able to accurately predict the state of the game.

Other possible preprocessing techniques could be ... ???

\subsection{Reward Clipping}

Different environments in the Atari gym have different reward structures, which means that training on the different games would mean that we need different hyper parameters and learning rates . To reduce this effect, we clip all non-zero rewards to either $1$, or $-1$. Of course, this will have a negative effect on the agent, since it cannot differentiate between small and big rewards. We assume that this effect is not too big\cite{mnih2015human}.


\subsection{Replay Memory}
Om patronen te voorkomen \cite{mnih2013playing}

Since we are playing the game on real time in the atari environment, there will be a high correlation between succesive game-states. Which might lead the algorithm to get stuck in a negative feedback loop. Consider the scenario in which moving to the right in the beginning of the game gives an immediate positive reward, but makes the player get stuck in some ???!!!???

In order to mitigate this effect, we use experience replay \cite{mnih2013playing}. We let the network gather some experience, before we start training the network; we let the network play the game, and we save the current game state, the taken action, the resulting reward and the resulting game state to a list of experiences. After the network has gained enough experience, we start training the network. Training is done by uniformly taking experiences from the experience list, and calculating the necessary things to train the network. More details on this will be provided in  chapter \ref{chapter:training}.

\subsection{Double DQN}

abc

\subsection{Training details} \label{chapter:training}
model initializatie + parameters etc. (qua hidden layers e.d.)
Huber loss vs. MSE

The model is trained by utilizing the techniques mentioned before. The network will be trained to approximate the $Q^*$ function. Using the simple loss function:

\[
	L = \left(r + \gamma \max_{a'}( Q(\theta_i, s',a')) - Q(\theta_i, s, a)\right)^2
\]

in which $\theta_i$ are the current weights, $s'$ is the state that is obtained after performing $a$ on the state $s$, $r$ is the reward obtained by performing action $a$ on state $s$.


\subsubsection{Training}
Start the training by initializing the necessary things: initialize the weights of the network, initialize the experience list, initialize the Atari environment and set the current screen to be all zeros. After that, we can start training the network. 

Then we loop for some number of entire games over the entire next phase of the training. We use the network to decide the actions taken in the game. If a game is finished, we simply reset the environment and continue training, we also have to reinitialize the current game state. Every time an action is taken, we collect the current state, the action, the reward, and the next state and store these in the experience list. After we have gathered sufficient experience, we start training on this experience. This is done by sampling random experiences from the list, and using the values from the experience to calculate the loss, and do backpropagation with this loss, to estimate how we have to update the network. 





% is it useful to make some pseudocode
%\begin{algorithm}
%	\caption{My algorithm}\label{euclid}
%	\begin{algorithmic}
%\STATE f
%\end{algorithmic}
%\end{algorithm}
%




\section{Results}


\section{Conclusion}

We can conclude that implement

\section{Discussion}
Complexer dan we dachten + link naar PNN
\subsection{Difficulties}
\subsection{PNN/ future work}
\cite{*}
\bibliography{bibliography}
\bibliographystyle{plainnat}

\end{document}
