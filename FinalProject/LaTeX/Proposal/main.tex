\documentclass{article}
\usepackage{a4wide}
\usepackage[utf8]{inputenc}
\usepackage[table,xcdraw]{xcolor}
\usepackage{ltablex} % table stretch multiple pages
\usepackage{booktabs} % Fancy table lines
\usepackage[numbers]{natbib}
\usepackage{graphicx}
\usepackage{enumerate}
\usepackage{array}
\usepackage[]{mathtools}
\usepackage{textcomp}
\usepackage{gensymb}
\usepackage{float}
\usepackage{amsmath}
\usepackage{dsfont}
\usepackage{csquotes}
\newcolumntype{b}{X}
\newcolumntype{s}{>{\hsize=.5\hsize}X}
\usepackage{pdfpages}
\usepackage{hyperref} % For links in documents and to the web

\newcommand{\q}[1]{``#1''} %matching apostrophes

\title{Project Proposal \\ Progressive Neural Networks}
\author{Dennis Verheijden s4455770 \and Joost Besseling s4796799}

\begin{document}
\maketitle
\section*{Project Description}
One of the many unsolved problems of Artificial Intelligence is catastrophic forgetting, the knowledge of task \textit{A} is forgotten when trained on task \textit{B}. This is what we aim to address in this project.

We will do so by implementing a Neural Network architecture proposed by Google Deepmind in 2016. Here they proposes a novel way to tackle this problem by using a Neural Network architecture that they called a Progressive Neural Network (PNN) \cite{rusu2016progressive}. For our project we will be implementing this type of network and we will train it to play atari games, using the openAi gym \cite{1606.01540}.

We plan on implementing it in Chainer \cite{chainer_learningsys2015}. With a network similar in the original paper, where they use two convolutional layers and one fully connect layer \cite{1606.01540}. The latter being tuned toward the task domain. The input to the neural networks are the raw pixel values of the rendered openAI scene.

We aim to replicate some of the papers findings by:
\begin{enumerate}[1.]
\item Implement a Progressive Neural Network
\item Train it on a source domain, which will be a Atari game (Pong probably)
\item Then train it on Pong variants and find out whether we can replicate the papers findings:
    \begin{itemize}
    \item Does the Progressive Neural Network converge faster than a regular Neural Network (i.e. is there transfer of knowledge)
    \item Try to compute the Average Pertubation Sensitivity to replicate the transfer matrix.
    \end{itemize}
\end{enumerate}

\bibliography{bibliography}
\bibliographystyle{plainnat}

\end{document}